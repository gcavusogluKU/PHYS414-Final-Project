\documentclass[aps,twocolumn,showpacs,preprintnumbers,nofootinbib,prl,superscriptaddress,groupedaddress]{revtex4-2}

% packages
\usepackage{amssymb,graphicx}
\usepackage{amsmath}
\usepackage{multirow}
\usepackage{epsfig}
\usepackage[usenames]{color} 
\usepackage[export]{adjustbox}
\usepackage{mathtools}
\usepackage{hyperref}

% you can define your own shortcuts etc, if you use sth very often
\newcommand{\balign}{\begin{align}}
\newcommand{\ealign}{\end{align}}

% or you can define your own symbols, check what this is
\def\meff{m_{\textrm{eff}}}


% here is where the document begins
\begin{document}

\title{PHYS 414 Final Project}
\author{G\"{o}rkem \c{C}avu\c{s}o\u{g}lu} % turkish characters require special handling
\affiliation{Department of Physics, Ko\c{c} University, \\
Rumelifeneri Yolu, 34450 Sariyer, Istanbul, Turkey }
\date{\today}

\begin{abstract}
\end{abstract}
\maketitle


%%%%%%%%%%%%%%%%%%%%%%%%%%%%%%%%%%%%%%%%
\section{Newton}
%%%%%%%%%%%%%%%%%%%%%%%%%%%%%%%%%%%%%%%%
%%%%%%%%%%%%%%%%%%%%%%%%%%%%%%%%%%%%%%%%


\subsection{(a) Analytical Derivations of Lane-Emden Equation}
Using Newtonian mechanics, two hydrostatic equilibrium equations are provided as
\begin{equation}\label{eq1}
	\frac{d m(r)}{dr}= 4\pi r^2 \rho(r)
\end{equation}
\begin{equation}\label{eq2}
	\frac{dp(r)}{dr} = -\frac{Gm(r)\rho(r)}{r^2}
\end{equation}
After rearranging terms of equation \ref{eq2} such that only $m(r)$ appears at right-hand side as a function of $r$, we obtain
\begin{equation}
	\frac{r^2}{\rho(r)}\frac{dp(r)}{dr} = -Gm(r)
\end{equation}
Taking derivatives of both sides, and using equation \ref{eq1} yield
\begin{equation}
	\frac{d}{dr}(\frac{r^2}{\rho(r)}\frac{dp(r)}{dr}) = -G\frac{dm(r)}{dr}= -4\pi G r^2 \rho(r)
\end{equation}
Thus, the equation is free of $m(r)$. Dividing both sides by $r^2$,
\begin{equation}
	\frac{1}{r^2}\frac{d}{dr}(\frac{r^2}{\rho(r)}\frac{dp(r)}{dr}) = -4\pi G  \rho(r)
\end{equation}
Using polytropic EOS $p = K\rho^{1+\frac{1}{n}}$, along with factorization $\rho(r) = \rho_c\theta^n(r)$ such that $p(r) = 
K\rho_c^{1+\frac{1}{n}}\theta^{n+1}(r)$,

\begin{align}
	\frac{K\rho_c^{\frac{1}{n}}}{r^2}\frac{d}{dr}(\frac{r^2}{\theta^n(r)}\frac{d}{dr}\theta^{n+1}(r)) = -4\pi G  \rho_c\theta^n(r) \\
	\frac{(n+1)K\rho_c^{\frac{1}{n}-1}}{4\pi G  }\frac{1}{r^2}\frac{d}{dr}(r^2\frac{d\theta(r)}{dr}) = -\theta^n(r)
\end{align}
If we use scaled radius such that $r = \alpha \xi$  where $\alpha^2 = \frac{(n+1)K\rho_c^{\frac{1}{n}-1}}{4\pi G  }$, we obtain the final form
\begin{equation}
	\frac{1}{\xi^2}\frac{d}{d\xi}(\xi^2\frac{d\theta(\xi)}{d\xi}) + \theta^n(\xi) = 0
\end{equation}

Using \texttt{Mathematica}'s asymptotic solve function \texttt{AsymptoticDSolveValue} to obtain a series solution of Lane-Emden equation, we obtain
\begin{equation}
\theta(\xi) \approx 1 - \frac{\xi^2}{6}+\frac{n\xi^4}{120}-\frac{(8n^2-5n)\xi^6}{1520}+\dots
\end{equation}

After this step, we can find closed-form analytical solutions of Lane-Emden equation. Evaluating Lane-Emden equation with \texttt{Mathematica}'s \texttt{DSolve} function for $n=1$ yields

\begin{equation}
	\theta(\xi) = \frac{sin(\xi)}{\xi}
\end{equation}


Since $M = \int_0^R 4\pi r^2\rho(r)dr = \frac{4\pi \rho_c}{\alpha^3}\int_0^{\xi_n} \xi^2\theta^n(\xi)d\xi$, substituting $\theta^n$ from Lane-Emden equation yields,
\begin{equation}
	M = 4\pi \rho_c\alpha^3\int_0^{\xi_n} -\frac{d}{d\xi}(\xi^2\frac{d\theta(\xi)}{d\xi})d\xi
\end{equation}
Evaluating the integral, and substituting $R = \alpha\xi_n$,
\begin{align}\label{eq12}{eq13}
&	M = 4\pi \rho_c\alpha^3[-\xi^2\frac{d\theta(\xi)}{d\xi}]\bigg|_0^{\xi_n} = 4\pi \rho_c\alpha^3\xi^3_n[-\frac{\theta'(\xi_n)}{\xi_n}] & \\
	&= 4\pi \rho_cR^3[-\frac{\theta'(\xi_n)}{\xi_n}] & 
	% \nonumber forces that line to have no reference number
	% the ampersand (&) is used to align the lines. the places you put them are at the same position in each line.
\end{align}
For the stars sharing same polytropic EOS, $\frac{R}{\alpha\xi_n}=1$, then
\begin{equation}
	\frac{4\pi G}{(n+1)K \xi_n^2}\rho_c^{\frac{n-1}{n}}R^2 =1
\end{equation}
Since $K$, $n$, $\xi_n$ and others are constant for same polytropic equation except $\rho_c$, $\rho_c \propto R^{\frac{-2n}{n-1}}$ . \\
\begin{equation}
	M \propto \rho_cR^3 \to M \propto R^{3-\frac{2n}{n-1}}=R^{\frac{3-n}{1-n}}
\end{equation}

\newpage
\subsection{(b) Mass versus Radius Distributions of White Dwarfs}

Using \texttt{Python} for reading and plotting the given white dwarf, we obtain mass distributions as a function of radii. After calculating corresponding radii \texttt{R} for data, the distribution is plotted. 

\begin{figure}[!htb]
	\centering
	\includegraphics[width=0.7\linewidth]{Plots/newton-part-b}
	\caption{}
	\label{fig:newton-part-b}
\end{figure}


\subsection{(c) Obtaining Fitting Parameters for White Dwarf Data}

Given pressure equation for the white dwarfs,

\begin{equation}
	C(x( 2x^2 - 3)(x^2 + 1)^{-1/2} + 3sinh^{-1}(x))
\end{equation}

where $x$ is the scaled density parameter $(\frac{\rho}{D})^{\frac{1}{q}}$. After obtaining series expansion for the pressure as $x\to 0$ using \texttt{Mathematica}'s \texttt{Series} function, we get the leading parameter to approximate pressure for small $x$

\begin{equation}
	P \approx \frac{8C}{5}(\frac{\rho}{D})^{\frac{5}{q}}
\end{equation}

After arranging the leading term in the form $P \approx K_* \rho^{1+\frac{1}{n_*}}$, we obtain parameters $K_*$ and $n_*$ as,

\begin{align}
	n_* = \frac{q}{5-q} & \quad\quad\quad \And & K_* \frac{8C}{5D^{\frac{5}{q}}}
\end{align}

After fitting the white dwarf data in \texttt{Python} for integer $q$, we obtain $q=3$ and using fitting parameters, we can plot the curve for small $M$

\begin{figure}[!htb]
	\centering
	\includegraphics[width=0.5\linewidth]{Plots/newton-part-c1}
	\caption{}
	\label{fig:newton-part-c1}
\end{figure}

Numerically solving Lane-Emden equation in \texttt{Python}, we obtain the solution for $n=\frac{3}{2}$ given below 

\begin{figure}[!htb]
	\centering
	\includegraphics[width=0.5\linewidth]{Plots/newton-part-c2}
	\caption{}
	\label{fig:newton-part-c2}
\end{figure}


By calculating the central densities of white dwarfs using \ref{eq13} and numerical solutions of Lane-Emden equation, we obtain the plot

\begin{figure}[!htb]
	\centering
	\includegraphics[width=0.5\linewidth]{Plots/newton-part-c3}
	\caption{}
	\label{fig:newton-part-c3}
\end{figure}

After fitting the data for parameter $K$, we obtain $K = 2774995.74$ and using the parameters, we plot the fitting curve

\begin{figure}[!htb]
	\centering
	\includegraphics[width=0.5\linewidth]{Plots/newton-part-c4}
	\caption{}
	\label{fig:newton-part-c4}
\end{figure}


\subsection{(d) Obtaining Parameter D by Interpolation}

Using interpolation and solving IVPs using differential equations $\frac{dm}{dr} = 4\pi r^2 \rho$ and,

\begin{equation}
	\frac{d\rho}{dr} = -G \frac{\sqrt{x^2+1}}{8Cx^5}\frac{qm\rho^2}{r^2}
\end{equation}

we obtained that $D = 3022830886$ and C = $10960543496614858194944$.

\subsection{(e)}

Using parameters, numerical solution of white dwarf mass-radius relation is plotted as

\begin{figure}[!htb]
	\centering
	\includegraphics[width=0.5\linewidth]{Plots/newton-part-e}
	\caption{}
	\label{fig:newton-part-e}
\end{figure}




\end{document}
